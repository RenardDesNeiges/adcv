%!TEX TS-program = xelatex
\documentclass[a4paper,online]{adcv}

\usepackage[french]{babel}
\usepackage{booktabs}
\usepackage{graphicx}
\usepackage{multirow}
\usepackage[export]{adjustbox}

\title{Titouan Renard's CV}

\adcvname{Titouan}{Renard}{}
\adcvtitle{Etudiant en master de Robotique avec un mineur de Data-Science}
\adcvaddress{}{}{Rue de la paix 9, 1020 Renens}{Suisse}
\adcvemail{titouan.renard}{epfl}{ch}
\adcvphone{(+41) 79 944 61 65}
\adcvwebsite{github}{github.com/RenardDesNeiges}
\adcvdate{July 2021}

\begin{document}

Etudiant en master de robotique à l'EPFL, je consacre une large partie de mon temps à faire avancer les questions de durabilité sur les campus UNIL/EPFL à travers mon engagement dans l'association Unipoly.

\section{Education}

\begin{adcvtabletwo}
  \adcvrowtwo{{En cours} \textbf{Master in Robotics with a Minor in Data Science}, Lausanne, Suisse}{2020-Présent}
  \adcvrowmulti{EPFL, facultés STI et IC}
  \adcvrowskip
  \adcvrowtwo{\textbf{Bachelor en Microtechnique}, Lausanne, Suisse}{2020}
  \adcvrowmulti{EPFL, faculté STI}
  \adcvrowskip
  \adcvrowtwo{\textbf{Maturité Bilingue}, Neuchâtel, Suisse}{2016}
  \adcvrowmulti{Bilingue anglais-français, Lycée Denis de Rougemont, Physique et Applications des Math}
\end{adcvtabletwo}

\section{Expérience}

\begin{adcvtabletwo}
  \adcvrowtwo{\textbf{Coprésident}, association Unipoly (association pour la durabilité sur les campus EPFL et UNIL).}{2021-2022}
  \adcvrowmulti{Responsable d'une association d'environ 300 membres avec un budget annuel d'environ 20'000 CHF.}
  \adcvrowskip
  \adcvrowtwo{\textbf{Assistant étudiant à l'EPFL}, pour les cours "Physique Générale 1", "Microinformatique" et "Science des Materiaux"}{2018-Present} 
  \adcvrowskip
  \adcvrowtwo{\textbf{Responsable Politique}, membre du comité de direction d'Unipoly, reponsable des interactions avec la direction de l'EPFL et les autres associations sur les questions de durabilité sur le campus.}{2020-2021}
  \adcvrowmulti{Experience de relations publique, contact avec les medias et communiqués de presse.}
  \adcvrowskip
  \adcvrowtwo{\textbf{Responsable de la programmation des évenements lors de la semaine de la durabilité 2020}, Unipoly, gestion et coordination d'environ 30 évenements sur les campus de l'EPFL and et l'UNIL.}{2020}
  \adcvrowmulti{Contact avec les intervenant·e·s, gestion des budgets.}
  \adcvrowskip
  \adcvrowtwo{\textbf{Stage d'été}, Laboratoire de Birorobotique EPFL, implémentation d'un système de vision pour la robotique en C++ avec OpenCV.}{2018}
  \adcvrowskip
  \adcvrowtwo{\textbf{European Youth Parliament}, Suisse}{2015-2016} 
  \adcvrowmulti{Expression orale en anglais.}
\end{adcvtabletwo}  

\section{Compétences}

\begin{adcvtabletwo}
  \adcvrowtwo{\textbf{Programmation}, en C, C++, Javascript, Matlab, Python et Scala. Experience des libraries OpenCV, Scipy, CasAdi, Tensorflow et Pytorch, bonne maitrise du shell des systèmes Unix et de git. Bonne connaissance de Wordpress et LaTex.}
  \adcvrowskip
  \adcvrowtwo{\textbf{Mathématiques et Mathématiques Appliquées}, Probabilités, Algèbre Linéaire, Analyse, Conception et Analyse d'Algorithmes. Capacité d'abstraction et de formulation de problèmes concrets sous forme mathématique.}
  \adcvrowskip
  \adcvrowtwo{\textbf{Langues}, Français (langue maternelle), Anglais (Niveau C1), Allemand (niveau lycée, B2).} 
  \adcvrowskip
  \adcvrowtwo{\textbf{Soft Skills}, expression, écriture et gestion d'équipes, appris dans le contexte d'associations étudiantes. Experience de communication (gestion de réseaux sociaux), design de site webs, mise en page d'affiches et de prospectus avec la suite adobe et d'autres outils.}{}
\end{adcvtabletwo}  

\end{document}
